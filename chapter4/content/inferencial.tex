%% chapter4/content/inferencial.tex - Análise Inferencial

\section{Análise Inferencial}
\label{sec:inferencial}

[Apresente os resultados dos testes estatísticos realizados.]

\subsection{Teste de Hipóteses}
\label{subsec:hipoteses}

[Apresente os resultados dos testes de hipóteses.]

% Exemplo de figura - coloque a imagem em chapter4/images/
% \begin{figure}[H]
%     \centering
%     \includegraphics[width=0.8\textwidth]{grafico_resultados}
%     \caption{Resultados da análise}
%     \label{fig:resultados}
% \end{figure}

\subsection{Análise de Correlação}
\label{subsec:correlacao}

[Se aplicável, apresente a análise de correlação entre variáveis.]

\begin{table}[H]
    \centering
    \caption{Matriz de correlação}
    \label{tab:correlacao}
    \begin{tabular}{lccc}
        \toprule
        & \textbf{Var A} & \textbf{Var B} & \textbf{Var C} \\
        \midrule
        Variável A & 1.00 & & \\
        Variável B & 0.XX** & 1.00 & \\
        Variável C & 0.XX* & 0.XX & 1.00 \\
        \bottomrule
        \multicolumn{4}{l}{\small *p$<$0.05; **p$<$0.01}
    \end{tabular}
\end{table}

\subsection{Análise de Regressão}
\label{subsec:regressao}

[Se aplicável, apresente os resultados da análise de regressão.]
