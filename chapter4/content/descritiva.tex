%% chapter4/content/descritiva.tex - Análise Descritiva

\section{Análise Descritiva}
\label{sec:descritiva}

[Apresente a caracterização da amostra e estatísticas descritivas.]

\subsection{Caracterização da Amostra}
\label{subsec:caracterizacao}

[Descreva as características demográficas ou outras características relevantes da amostra.]

% Exemplo de tabela - crie em chapter4/tables/
% \input{chapter4/tables/tabela_amostra.tex}

% Exemplo de tabela inline:
\begin{table}[H]
    \centering
    \caption{Caracterização da amostra}
    \label{tab:amostra}
    \begin{tabular}{lcc}
        \toprule
        \textbf{Característica} & \textbf{n} & \textbf{\%} \\
        \midrule
        Género & & \\
        \quad Masculino & XX & XX\% \\
        \quad Feminino & XX & XX\% \\
        \midrule
        Faixa Etária & & \\
        \quad 18-25 anos & XX & XX\% \\
        \quad 26-35 anos & XX & XX\% \\
        \quad $>$35 anos & XX & XX\% \\
        \bottomrule
    \end{tabular}
\end{table}

\subsection{Estatísticas Descritivas}
\label{subsec:estatisticas}

[Apresente as estatísticas descritivas das variáveis em estudo.]

\begin{table}[H]
    \centering
    \caption{Estatísticas descritivas das variáveis}
    \label{tab:descritivas}
    \begin{tabular}{lccccc}
        \toprule
        \textbf{Variável} & \textbf{N} & \textbf{Mín.} & \textbf{Máx.} & \textbf{Média} & \textbf{DP} \\
        \midrule
        Variável A & XX & X.XX & X.XX & X.XX & X.XX \\
        Variável B & XX & X.XX & X.XX & X.XX & X.XX \\
        Variável C & XX & X.XX & X.XX & X.XX & X.XX \\
        \bottomrule
    \end{tabular}
\end{table}
