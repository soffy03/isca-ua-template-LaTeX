%% resumo.tex - Agradecimentos, Resumo e Abstract
%% Edite este ficheiro com os seus textos

%% ============================================================================
%% AGRADECIMENTOS (opcional)
%% ============================================================================
\begin{agradecimentos}
Agradeço primeiramente ao meu orientador, Prof. Doutor [Nome], pela dedicação, disponibilidade e valiosas orientações ao longo deste trabalho.

Agradeço também à minha família pelo apoio incondicional durante todo o percurso académico.

Aos meus colegas e amigos que me acompanharam nesta jornada.

[Adicione outros agradecimentos conforme necessário]
\end{agradecimentos}

%% ============================================================================
%% RESUMO EM PORTUGUÊS
%% ============================================================================
\begin{resumo}
[Escreva aqui o resumo da sua dissertação em português. O resumo deve ser um único parágrafo que sintetize os objetivos, metodologia, principais resultados e conclusões do trabalho. Deve ter entre 150 a 300 palavras.]

Este trabalho tem como objetivo [objetivo principal]. Para tal, foi adotada uma metodologia [tipo de metodologia], tendo sido [descrição breve do método]. Os resultados obtidos demonstram que [principais conclusões]. Este estudo contribui para [contribuição para a área].
\end{resumo}

%% ============================================================================
%% ABSTRACT EM INGLÊS
%% ============================================================================
\begin{abstracten}
[Write here the abstract of your dissertation in English. The abstract should be a single paragraph that summarizes the objectives, methodology, main results and conclusions of the work. It should have between 150 and 300 words.]

This work aims to [main objective]. To this end, a [type of methodology] methodology was adopted, having been [brief description of the method]. The results obtained demonstrate that [main conclusions]. This study contributes to [contribution to the field].
\end{abstracten}
