%% main.tex - Template ISCA-UA (Estrutura Modular)
%% Instituto Superior de Contabilidade e Administração - Universidade de Aveiro
%% 
%% Normas oficiais: https://www.ua.pt/pt/sga/page/12810
%% Cores ISCA/Contabilidade: Pantone 301 (Azul) + Pantone 1787 (Vermelho)

\documentclass[12pt,a4paper,twoside,openright]{book}

%% ============================================================================
%% CONFIGURAÇÃO DO TEMPLATE
%% ============================================================================
\usepackage{config/config}

%% ============================================================================
%% BIBLIOGRAFIA (escolha o estilo adequado à sua área)
%% ============================================================================
\usepackage[style=apa,backend=biber]{biblatex}
% Alternativas:
% \usepackage[style=ieee,backend=biber]{biblatex}
% \usepackage[style=numeric,backend=biber]{biblatex}

\addbibresource{biblio.bib}

%% ============================================================================
%% DADOS DO DOCUMENTO (EDITAR AQUI)
%% ============================================================================
\titulo{Título da Dissertação em Português}
\tituloingles{Dissertation Title in English}

\autor{Nome Completo do Autor}

\curso{Contabilidade}

\departamento{Instituto Superior de Contabilidade e Administração}

\orientador{Prof. Doutor Nome do Orientador, Professor Associado do ISCA-UA}
\coorientador{Prof. Doutor Nome do Coorientador, Professor Auxiliar do ISCA-UA}

\ano{2025}

\grau{Mestre}

\palavraschave{palavra1, palavra2, palavra3, palavra4, palavra5}
\keywords{keyword1, keyword2, keyword3, keyword4, keyword5}

\declaracaoia{%
Na elaboração desta dissertação foi utilizada/não foi utilizada Inteligência Artificial generativa. 
[Se foi utilizada:] As ferramentas utilizadas foram [nome das ferramentas] para [finalidade].
O autor assume total responsabilidade pelo conteúdo final do trabalho.
}

%% ============================================================================
%% MEMBROS DO JÚRI (preencher após nomeação)
%% ============================================================================
\juripresidente{Prof. Doutor Nome Completo\\
\small Professor Catedrático da Universidade de Aveiro}

\jurivogal{Prof. Doutor Nome Completo\\
\small Professor Associado da [Instituição]}

\juriorientador{Prof. Doutor Nome Completo\\
\small Professor [Categoria] do ISCA-UA}

%% ============================================================================
%% INÍCIO DO DOCUMENTO
%% ============================================================================
\begin{document}

%% ----------------------------------------------------------------------------
%% ELEMENTOS PRÉ-TEXTUAIS (numeração romana)
%% ----------------------------------------------------------------------------
\frontmatter

\fazercapa
\fazerfolharosto
\fazerpaginajuri

% Agradecimentos, Resumo, Abstract
%% resumo.tex - Agradecimentos, Resumo e Abstract
%% Edite este ficheiro com os seus textos

%% ============================================================================
%% AGRADECIMENTOS (opcional)
%% ============================================================================
\begin{agradecimentos}
Agradeço primeiramente ao meu orientador, Prof. Doutor [Nome], pela dedicação, disponibilidade e valiosas orientações ao longo deste trabalho.

Agradeço também à minha família pelo apoio incondicional durante todo o percurso académico.

Aos meus colegas e amigos que me acompanharam nesta jornada.

[Adicione outros agradecimentos conforme necessário]
\end{agradecimentos}

%% ============================================================================
%% RESUMO EM PORTUGUÊS
%% ============================================================================
\begin{resumo}
[Escreva aqui o resumo da sua dissertação em português. O resumo deve ser um único parágrafo que sintetize os objetivos, metodologia, principais resultados e conclusões do trabalho. Deve ter entre 150 a 300 palavras.]

Este trabalho tem como objetivo [objetivo principal]. Para tal, foi adotada uma metodologia [tipo de metodologia], tendo sido [descrição breve do método]. Os resultados obtidos demonstram que [principais conclusões]. Este estudo contribui para [contribuição para a área].
\end{resumo}

%% ============================================================================
%% ABSTRACT EM INGLÊS
%% ============================================================================
\begin{abstracten}
[Write here the abstract of your dissertation in English. The abstract should be a single paragraph that summarizes the objectives, methodology, main results and conclusions of the work. It should have between 150 and 300 words.]

This work aims to [main objective]. To this end, a [type of methodology] methodology was adopted, having been [brief description of the method]. The results obtained demonstrate that [main conclusions]. This study contributes to [contribution to the field].
\end{abstracten}


% Declaração IA
\fazerdeclaracaoia

% Índices (aparecem antes dos acrónimos)
\tableofcontents
\listoffigures
\listoftables

% Acrónimos e Glossário
%% acronimos.tex - Lista de Acrónimos e Siglas
%% Adicione os acrónimos utilizados no seu trabalho

\cleardoublepage
\chapter*{Lista de Acrónimos e Siglas}
\addcontentsline{toc}{chapter}{Lista de Acrónimos e Siglas}

\begin{tabular}{ll}
    \textbf{ISCA-UA} & Instituto Superior de Contabilidade e Administração de Aveiro \\[0.3em]
    \textbf{UA} & Universidade de Aveiro \\[0.3em]
    \textbf{SNC} & Sistema de Normalização Contabilística \\[0.3em]
    \textbf{IAS} & International Accounting Standards \\[0.3em]
    \textbf{IFRS} & International Financial Reporting Standards \\[0.3em]
    \textbf{NCRF} & Normas Contabilísticas e de Relato Financeiro \\[0.3em]
    \textbf{IRC} & Imposto sobre o Rendimento das Pessoas Coletivas \\[0.3em]
    \textbf{IVA} & Imposto sobre o Valor Acrescentado \\[0.3em]
    \textbf{PME} & Pequenas e Médias Empresas \\[0.3em]
    % Adicione mais acrónimos conforme necessário
    % \textbf{SIGLA} & Significado da sigla \\[0.3em]
\end{tabular}

%% glossario.tex - Glossário de Termos
%% Adicione definições de termos técnicos utilizados no trabalho
%% (Este ficheiro é opcional - comente o \input no main.tex se não precisar)

\cleardoublepage
\chapter*{Glossário}
\addcontentsline{toc}{chapter}{Glossário}

\begin{description}[leftmargin=3cm, style=nextline]
    
    \item[Ativo] Recurso controlado pela entidade como resultado de acontecimentos passados e do qual se espera que fluam para a entidade benefícios económicos futuros.
    
    \item[Passivo] Obrigação presente da entidade proveniente de acontecimentos passados, cuja liquidação se espera que resulte num exfluxo de recursos da entidade.
    
    \item[Capital Próprio] Interesse residual nos ativos da entidade depois de deduzidos todos os seus passivos.
    
    \item[Rendimento] Aumentos nos benefícios económicos durante o período contabilístico na forma de influxos ou aumentos de ativos ou diminuições de passivos.
    
    \item[Gasto] Diminuições nos benefícios económicos durante o período contabilístico na forma de exfluxos ou deperecimentos de ativos ou na incorrência de passivos.
    
    % Adicione mais termos conforme necessário
    % \item[Termo] Definição do termo.
    
\end{description}


%% ----------------------------------------------------------------------------
%% CORPO DO DOCUMENTO (numeração árabe, começa em 1)
%% ----------------------------------------------------------------------------
\mainmatter

% Capítulo 1 - Introdução
%% chapter1/include.tex - Capítulo 1: Introdução
%% Este ficheiro carrega todo o conteúdo do capítulo

\chapter{Introdução}
\label{cap:introducao}

%% Carrega as secções do capítulo
%% chapter1/content/enquadramento.tex - Enquadramento e Motivação

\section{Enquadramento e Motivação}
\label{sec:enquadramento}

[Descreva o contexto em que se insere o seu trabalho e a motivação para a sua realização.]

O presente trabalho insere-se na área de [área de estudo], um tema de crescente relevância no contexto atual devido a [razões de relevância].

A motivação para este estudo surge da necessidade de [necessidade identificada], uma vez que [justificação].

% Exemplo de citação (descomente e adapte):
% Segundo \textcite{autor2023}, esta problemática tem vindo a ganhar destaque nos últimos anos.

%% chapter1/content/objetivos.tex - Objetivos

\section{Objetivos}
\label{sec:objetivos}

[Apresente os objetivos gerais e específicos do seu trabalho.]

\subsection{Objetivo Geral}
\label{subsec:objetivo_geral}

O objetivo geral deste trabalho é [descrever o objetivo principal do estudo].

\subsection{Objetivos Específicos}
\label{subsec:objetivos_especificos}

Para alcançar o objetivo geral, foram definidos os seguintes objetivos específicos:

\begin{itemize}
    \item Identificar [objetivo específico 1];
    \item Analisar [objetivo específico 2];
    \item Avaliar [objetivo específico 3];
    \item Propor [objetivo específico 4].
\end{itemize}

\subsection{Questões de Investigação}
\label{subsec:questoes}

Com base nos objetivos definidos, formularam-se as seguintes questões de investigação:

\begin{enumerate}
    \item [Questão de investigação 1]?
    \item [Questão de investigação 2]?
    \item [Questão de investigação 3]?
\end{enumerate}

%% chapter1/content/estrutura.tex - Estrutura da Dissertação

\section{Estrutura da Dissertação}
\label{sec:estrutura}

A presente dissertação está organizada em cinco capítulos, estruturados da seguinte forma:

O \textbf{Capítulo~\ref{cap:introducao}} apresenta a introdução do trabalho, incluindo o enquadramento e motivação, os objetivos definidos e a estrutura da dissertação.

O \textbf{Capítulo~\ref{cap:revisao}} dedica-se à revisão da literatura, onde são apresentados os conceitos fundamentais e o estado da arte na área de estudo.

O \textbf{Capítulo~\ref{cap:metodologia}} descreve a metodologia adotada, incluindo a abordagem metodológica, os métodos de recolha e análise de dados.

O \textbf{Capítulo~\ref{cap:resultados}} apresenta e discute os resultados obtidos no estudo.

Por fim, o \textbf{Capítulo~\ref{cap:conclusoes}} apresenta as principais conclusões do trabalho, as limitações identificadas e sugestões para investigação futura.



% Capítulo 2 - Revisão da Literatura
%% chapter2/include.tex - Capítulo 2: Revisão da Literatura

\chapter{Revisão da Literatura}
\label{cap:revisao}

Neste capítulo apresenta-se a revisão bibliográfica sobre o tema em estudo, abordando os conceitos fundamentais e o estado da arte.

%% Carrega as secções do capítulo
%% chapter2/content/conceitos.tex - Conceitos Fundamentais

\section{Conceitos Fundamentais}
\label{sec:conceitos}

[Apresente os conceitos teóricos fundamentais para o seu trabalho.]

\subsection{[Conceito 1]}
\label{subsec:conceito1}

[Definição e explicação do primeiro conceito relevante.]

\subsection{[Conceito 2]}
\label{subsec:conceito2}

[Definição e explicação do segundo conceito relevante.]

\subsection{Enquadramento Normativo}
\label{subsec:normativo}

[Se aplicável, apresente o enquadramento normativo/legal relevante para o seu estudo.]

% Exemplo de tabela (a partir do ficheiro tables)
% \input{chapter2/tables/tabela_normas.tex}

%% chapter2/content/estadoarte.tex - Estado da Arte

\section{Estado da Arte}
\label{sec:estadoarte}

[Descreva os trabalhos relacionados e o estado atual do conhecimento na área.]

\subsection{Estudos Anteriores}
\label{subsec:estudos}

[Apresente uma revisão dos estudos mais relevantes na área.]

% Exemplo de citação:
% O estudo de \textcite{autor2022} demonstrou que...
% Esta conclusão é corroborada por \textcite{outro2021}.

\subsection{Síntese da Literatura}
\label{subsec:sintese}

[Faça uma síntese crítica da literatura revista, identificando lacunas e oportunidades de investigação.]

% Exemplo de tabela comparativa de estudos
% \input{chapter2/tables/tabela_estudos.tex}



% Capítulo 3 - Metodologia
%% chapter3/include.tex - Capítulo 3: Metodologia

\chapter{Metodologia}
\label{cap:metodologia}

Neste capítulo descreve-se a metodologia utilizada no desenvolvimento do trabalho, incluindo a abordagem metodológica, os métodos de recolha e análise de dados.

%% Carrega as secções do capítulo
%% chapter3/content/abordagem.tex - Abordagem Metodológica

\section{Abordagem Metodológica}
\label{sec:abordagem}

[Descreva a abordagem metodológica escolhida.]

Este estudo adota uma abordagem [quantitativa/qualitativa/mista], considerando [justificação da escolha].

\subsection{Tipo de Estudo}
\label{subsec:tipo_estudo}

Quanto ao tipo de estudo, esta investigação classifica-se como [exploratório/descritivo/explicativo/correlacional], uma vez que [justificação].

\subsection{População e Amostra}
\label{subsec:amostra}

[Descreva a população-alvo e o processo de amostragem utilizado.]

A população deste estudo é constituída por [descrição da população]. A amostra foi selecionada através de [método de amostragem], resultando em [tamanho da amostra] elementos.

%% chapter3/content/recolha.tex - Recolha de Dados

\section{Recolha de Dados}
\label{sec:recolha}

[Explique como foram recolhidos os dados.]

\subsection{Instrumentos de Recolha}
\label{subsec:instrumentos}

Para a recolha de dados foram utilizados os seguintes instrumentos:

\begin{itemize}
    \item [Instrumento 1]: [descrição e justificação];
    \item [Instrumento 2]: [descrição e justificação].
\end{itemize}

\subsection{Procedimentos}
\label{subsec:procedimentos}

[Descreva os procedimentos de recolha de dados, incluindo período, forma de contacto, etc.]

\subsection{Considerações Éticas}
\label{subsec:etica}

[Se aplicável, descreva as considerações éticas do estudo, como consentimento informado, anonimato, etc.]

%% chapter3/content/analise.tex - Análise de Dados

\section{Análise de Dados}
\label{sec:analise}

[Descreva os métodos de análise utilizados.]

\subsection{Técnicas de Análise}
\label{subsec:tecnicas}

Para a análise dos dados recolhidos foram utilizadas as seguintes técnicas:

\begin{itemize}
    \item \textbf{Análise descritiva}: [descrição];
    \item \textbf{Análise inferencial}: [descrição dos testes estatísticos, se aplicável];
    \item \textbf{Análise de conteúdo}: [se aplicável para dados qualitativos].
\end{itemize}

\subsection{Software Utilizado}
\label{subsec:software}

O tratamento e análise dos dados foi realizado com recurso ao software [nome do software, ex: SPSS, R, Excel, NVivo], versão [versão].

% Exemplo de figura do processo metodológico
% \begin{figure}[H]
%     \centering
%     \includegraphics[width=0.8\textwidth]{metodologia}
%     \caption{Esquema do processo metodológico}
%     \label{fig:metodologia}
% \end{figure}



% Capítulo 4 - Resultados
%% chapter4/include.tex - Capítulo 4: Resultados e Discussão

\chapter{Resultados e Discussão}
\label{cap:resultados}

Neste capítulo apresentam-se e discutem-se os resultados obtidos no estudo.

%% Carrega as secções do capítulo
%% chapter4/content/descritiva.tex - Análise Descritiva

\section{Análise Descritiva}
\label{sec:descritiva}

[Apresente a caracterização da amostra e estatísticas descritivas.]

\subsection{Caracterização da Amostra}
\label{subsec:caracterizacao}

[Descreva as características demográficas ou outras características relevantes da amostra.]

% Exemplo de tabela - crie em chapter4/tables/
% \input{chapter4/tables/tabela_amostra.tex}

% Exemplo de tabela inline:
\begin{table}[H]
    \centering
    \caption{Caracterização da amostra}
    \label{tab:amostra}
    \begin{tabular}{lcc}
        \toprule
        \textbf{Característica} & \textbf{n} & \textbf{\%} \\
        \midrule
        Género & & \\
        \quad Masculino & XX & XX\% \\
        \quad Feminino & XX & XX\% \\
        \midrule
        Faixa Etária & & \\
        \quad 18-25 anos & XX & XX\% \\
        \quad 26-35 anos & XX & XX\% \\
        \quad $>$35 anos & XX & XX\% \\
        \bottomrule
    \end{tabular}
\end{table}

\subsection{Estatísticas Descritivas}
\label{subsec:estatisticas}

[Apresente as estatísticas descritivas das variáveis em estudo.]

\begin{table}[H]
    \centering
    \caption{Estatísticas descritivas das variáveis}
    \label{tab:descritivas}
    \begin{tabular}{lccccc}
        \toprule
        \textbf{Variável} & \textbf{N} & \textbf{Mín.} & \textbf{Máx.} & \textbf{Média} & \textbf{DP} \\
        \midrule
        Variável A & XX & X.XX & X.XX & X.XX & X.XX \\
        Variável B & XX & X.XX & X.XX & X.XX & X.XX \\
        Variável C & XX & X.XX & X.XX & X.XX & X.XX \\
        \bottomrule
    \end{tabular}
\end{table}

%% chapter4/content/inferencial.tex - Análise Inferencial

\section{Análise Inferencial}
\label{sec:inferencial}

[Apresente os resultados dos testes estatísticos realizados.]

\subsection{Teste de Hipóteses}
\label{subsec:hipoteses}

[Apresente os resultados dos testes de hipóteses.]

% Exemplo de figura - coloque a imagem em chapter4/images/
% \begin{figure}[H]
%     \centering
%     \includegraphics[width=0.8\textwidth]{grafico_resultados}
%     \caption{Resultados da análise}
%     \label{fig:resultados}
% \end{figure}

\subsection{Análise de Correlação}
\label{subsec:correlacao}

[Se aplicável, apresente a análise de correlação entre variáveis.]

\begin{table}[H]
    \centering
    \caption{Matriz de correlação}
    \label{tab:correlacao}
    \begin{tabular}{lccc}
        \toprule
        & \textbf{Var A} & \textbf{Var B} & \textbf{Var C} \\
        \midrule
        Variável A & 1.00 & & \\
        Variável B & 0.XX** & 1.00 & \\
        Variável C & 0.XX* & 0.XX & 1.00 \\
        \bottomrule
        \multicolumn{4}{l}{\small *p$<$0.05; **p$<$0.01}
    \end{tabular}
\end{table}

\subsection{Análise de Regressão}
\label{subsec:regressao}

[Se aplicável, apresente os resultados da análise de regressão.]

%% chapter4/content/discussao.tex - Discussão dos Resultados

\section{Discussão dos Resultados}
\label{sec:discussao}

[Discuta os resultados obtidos à luz da literatura revista.]

\subsection{Interpretação dos Resultados}
\label{subsec:interpretacao}

Os resultados obtidos neste estudo [interpretação geral dos resultados].

Em relação ao primeiro objetivo, verificou-se que [discussão do resultado 1]. Este resultado está em consonância com os estudos de [autores], que também identificaram [resultado semelhante].

Relativamente ao segundo objetivo, os dados demonstram que [discussão do resultado 2]. Contrariamente ao esperado com base na literatura, [ou: Em linha com a literatura], este resultado pode ser explicado por [explicação].

\subsection{Implicações Teóricas}
\label{subsec:implicacoes_teoricas}

[Discuta as implicações teóricas dos resultados.]

\subsection{Implicações Práticas}
\label{subsec:implicacoes_praticas}

[Discuta as implicações práticas dos resultados para profissionais e organizações.]



% Capítulo 5 - Conclusões
%% chapter5/include.tex - Capítulo 5: Conclusões

\chapter{Conclusões}
\label{cap:conclusoes}

Neste capítulo apresentam-se as principais conclusões do trabalho, as limitações identificadas e sugestões para investigação futura.

%% Carrega as secções do capítulo
%% chapter5/content/sintese.tex - Síntese e Conclusões

\section{Síntese do Trabalho}
\label{sec:sintese}

[Sintetize os principais contributos do trabalho, respondendo às questões de investigação.]

Este trabalho teve como objetivo principal [objetivo]. Para tal, foi desenvolvido um estudo [tipo de estudo] que permitiu [o que foi alcançado].

\subsection{Resposta às Questões de Investigação}
\label{subsec:respostas}

Em resposta à primeira questão de investigação, ``[QI1]?'', os resultados demonstram que [resposta baseada nos resultados].

Relativamente à segunda questão, ``[QI2]?'', verificou-se que [resposta].

Quanto à terceira questão, ``[QI3]?'', os dados indicam que [resposta].

\subsection{Contributos do Estudo}
\label{subsec:contributos}

Este estudo contribui para [área] de várias formas:

\begin{itemize}
    \item [Contributo teórico 1];
    \item [Contributo prático 1];
    \item [Contributo metodológico, se aplicável].
\end{itemize}

%% chapter5/content/limitacoes.tex - Limitações do Estudo

\section{Limitações do Estudo}
\label{sec:limitacoes}

[Apresente as limitações do estudo de forma honesta e objetiva.]

Apesar dos contributos apresentados, este estudo apresenta algumas limitações que devem ser consideradas na interpretação dos resultados:

\begin{enumerate}
    \item \textbf{Limitação metodológica}: [descrição, ex: tamanho da amostra, método de amostragem];
    
    \item \textbf{Limitação temporal}: [descrição, ex: período de recolha de dados];
    
    \item \textbf{Limitação geográfica}: [descrição, ex: estudo limitado a uma região/setor];
    
    \item \textbf{Limitação dos dados}: [descrição, ex: disponibilidade de dados, variáveis não controladas].
\end{enumerate}

Estas limitações não invalidam os resultados obtidos, mas devem ser tidas em conta na sua generalização.

%% chapter5/content/futuro.tex - Investigação Futura

\section{Sugestões para Investigação Futura}
\label{sec:futuro}

[Sugira linhas de investigação futura que possam dar continuidade a este trabalho.]

Com base nas conclusões e limitações identificadas, sugerem-se as seguintes linhas de investigação futura:

\begin{itemize}
    \item \textbf{Ampliação do estudo}: Replicar este estudo com uma amostra maior e mais diversificada, permitindo [benefício esperado];
    
    \item \textbf{Estudo longitudinal}: Desenvolver um estudo de natureza longitudinal que permita avaliar [aspeto a avaliar ao longo do tempo];
    
    \item \textbf{Novas variáveis}: Incluir outras variáveis na análise, como [sugestão de variáveis], que podem influenciar [resultado];
    
    \item \textbf{Diferentes contextos}: Aplicar o estudo em diferentes contextos [geográficos/setoriais/organizacionais] para verificar a generalização dos resultados;
    
    \item \textbf{Abordagem metodológica complementar}: Complementar este estudo com uma abordagem [qualitativa/quantitativa] que permita [benefício].
\end{itemize}

\vspace{1cm}

\noindent\textit{Considera-se que este trabalho constitui um contributo válido para [área de estudo], abrindo caminho para futuras investigações que aprofundem o conhecimento nesta matéria.}



%% ----------------------------------------------------------------------------
%% REFERÊNCIAS BIBLIOGRÁFICAS
%% ----------------------------------------------------------------------------
\addbibliografia
\printbibliography

%% ----------------------------------------------------------------------------
%% ANEXOS
%% ----------------------------------------------------------------------------
\anexos
%% annexes/include.tex - Anexos
%% Adicione aqui os seus anexos

\chapter{Questionário Utilizado}
\label{anx:questionario}

[Inclua aqui o questionário ou outros instrumentos de recolha de dados utilizados.]

% Se tiver o questionário em PDF:
% \includepdf[pages=-]{annexes/questionario.pdf}

\vspace{1cm}
\noindent\textit{[Questionário a incluir]}


\chapter{Outputs Estatísticos Complementares}
\label{anx:outputs}

[Inclua aqui outputs estatísticos adicionais que não foram incluídos no corpo do texto mas que são relevantes para a compreensão do estudo.]

% Exemplo de tabela extensa
% \input{annexes/tabela_completa.tex}


\chapter{Documentos de Apoio}
\label{anx:documentos}

[Inclua aqui outros documentos de apoio, como autorizações, declarações de consentimento, etc.]

% Se tiver documentos em PDF:
% \includepdf[pages=-]{annexes/autorizacao.pdf}


\end{document}
