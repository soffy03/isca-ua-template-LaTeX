%% chapter5/content/futuro.tex - Investigação Futura

\section{Sugestões para Investigação Futura}
\label{sec:futuro}

[Sugira linhas de investigação futura que possam dar continuidade a este trabalho.]

Com base nas conclusões e limitações identificadas, sugerem-se as seguintes linhas de investigação futura:

\begin{itemize}
    \item \textbf{Ampliação do estudo}: Replicar este estudo com uma amostra maior e mais diversificada, permitindo [benefício esperado];
    
    \item \textbf{Estudo longitudinal}: Desenvolver um estudo de natureza longitudinal que permita avaliar [aspeto a avaliar ao longo do tempo];
    
    \item \textbf{Novas variáveis}: Incluir outras variáveis na análise, como [sugestão de variáveis], que podem influenciar [resultado];
    
    \item \textbf{Diferentes contextos}: Aplicar o estudo em diferentes contextos [geográficos/setoriais/organizacionais] para verificar a generalização dos resultados;
    
    \item \textbf{Abordagem metodológica complementar}: Complementar este estudo com uma abordagem [qualitativa/quantitativa] que permita [benefício].
\end{itemize}

\vspace{1cm}

\noindent\textit{Considera-se que este trabalho constitui um contributo válido para [área de estudo], abrindo caminho para futuras investigações que aprofundem o conhecimento nesta matéria.}
